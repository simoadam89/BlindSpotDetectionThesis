\documentclass[15pt]{scrartcl}
\usepackage[utf8]{inputenc}
\usepackage[T1]{fontenc}
\usepackage[ngerman]{babel}
\usepackage{graphicx}
\usepackage{float}
\usepackage[normalem]{ulem}
\setlength{\parindent}{1em}
\setlength{\parskip}{1em}
\renewcommand{\baselinestretch}{1.2}

\begin{document}
\thispagestyle{empty}
\quad  \addtocounter{page}{-1}

\title{Thesis}
\author{Mohamed El Ouadia}
\date{\today}

\begin{figure}[h]
\begin{minipage}{0.5\textwidth }
	\includegraphics[width=.65\linewidth]{htwlogo}	
\end{minipage}\hspace{5cm}
\begin{minipage}{0.5\textwidth}
	\includegraphics[width=.5\linewidth]{htw}
\end{minipage}
\end{figure}

\begin{center}
\begin{large}
\textbf{Bachelor-Thesis}\break 
\end{large}

\textbf{"Blind Spot Detection mittels V2x Kommunikation"}\break

\textbf {Bachelor of Science(B.Sc)} \break \break 

an der Hochschule für Technik und Wirtschaft des Saarlandes\break 
im Studiengang\break \textbf{ Kommunikationsinformatik}\break 

vorgelegt von \break 
Mohamed El Ouadia\break 
3603547\break 

Betreut von :\break 
Andreas Otte, M.Sc.\break 
Prof. Dr.-Ing. H. Wieker\break\break \break 
Saarbrücken, 19.09.2019
\end{center}

\newpage
\thispagestyle{empty}
\quad  \addtocounter{page}{-1}

\newpage
\thispagestyle{empty}
\quad  \addtocounter{page}{-1}
\begin{large}
\textbf{Eidestattliche Erklärung}
\end{large}

Hiermit versichere ich, 
Mohamed El Ouadia
geboren am 16.12.1989
in Beni Melall, dass ich die vorliegende Arbeit selbstständig verfasst und keine anderen als die angegebenen Quellen und Hilfsmittel benutzt habe, dass ich zuvor an keiner anderen Hochschule und in keinem anderen Studiangang als Prüfungsleistung eingereicht habe.
ich versichere weiterhin,dass alle Ausführungen, die anderen Schriften wörtlich oder sinngemäß entnommen wurden, kenntlich gemacht sind und die Arbeit in gleicher oder ähnlicher Fassung noch nicht Bestandteil einer Studien- oder Prüfungsleistung war.\newline

Saarbrücken 17.September 2019 \newline\newline

\begin{tabular}{@{}l@{}}\hline
\textsl{Mohamed El Ouadia}
\end{tabular}
\newpage
\thispagestyle{empty}
\quad  \addtocounter{page}{-1}
\newpage
\thispagestyle{empty}
\quad  \addtocounter{page}{-1}
\begin{large}
\textbf{Danksagung}
\end{large}

Ich möchte allen danken, die mich bei meinem Studium und bei der Erstellung dieser Bachelor-Thesis unterstützt haben und mir die Arbeit möglich gemacht haben. Mein Besonderes Dank geht an Prof. Dr.-Ing. Horst Wieker und an meinem Betreuer Andreas Otte, der immer offen für Fragen war und der mir immer mit Rat zur Seite Stand. 
Zudem möchte ich Fabian Coulet und allen Mitarbeiter der Forschungsgruppe Verkehrstelematik der htwsaar für das gute Arbeitsklima und deren Unterstützung in den Vergangenen Monaten Danken.

Des weiteren danke ich meine Mutter und meiner familie für die moralische und seelische Unterstützung.
\newpage
\thispagestyle{empty}
\quad  \addtocounter{page}{-1}

\newpage
\thispagestyle{empty}

\newpage
\tableofcontents
\newpage
\section{Einleitung}
\subsection{Vorwort}
 Die vorliegende Bachelor-Thesis entstand an der Hochschule für Technik und Wirtschaft des Saarlandes im Rahmen des Bachelor-Studiengangs Kommunikationsinformatik.
 Die Umsetzung der Arbeit fand in der Forschungsgruppe Verkehrstelematik (FGVT) statt, unter der Leitung vom Herrn
Prof. Dr.-Ing. Horst Wieker. 
Die Arbeit umfasst die Konzeption und die Entwicklung der Anwendung
BlindSpotDetection(BSD).
Die Anwendung soll Fahrzeuge im Toten Winkel erkennen und dem fahrer einen Information auf HMI übermitteln falls kein Gefahr steht, sobald die Situation gefährlich wird, ändert sich die Anzeige vom Information zu einer Warnung mit Hinweis auf Abstand und auf welcher Seite das fahrzeug erkannt wurde.
\subsection{Motivation}
 Ich bin motivation
\subsection{Über FGVT}
 FGVT ist ...
\subsection{ITeM}
 Das Forschungsprojekt CONVERGE (COmmuicationNetwork VEhicle   RoadGlobal Exten-
sion) 

Das Projekt iKoPA (Integrierte Kommunikationsplattform für automatisierte Elektrofahr-
zeuge) ist da um Infrastruktursysteme miteinander zu verbinden.



\newpage
\section{Grundlagen}
\subsection{Intelligent Transport Systems}
\subsection{V2x}
\subsection{CAM}
\subsection{OSGI}
\subsection{BlindSpotDetection}
\subsection{Anforderungen}
\subsection{PositionProvider}
\subsection{SensorDataHandler}
\subsection{LocalDynamicMap}
\subsection{StationInfo}
\subsection{Vorhandene Komponente}
\subsection{Use Cases}

\newpage
\section{Projektkonzept}
\subsection{Übersicht}
\subsection{Projektablauf}
\subsection{BlindSpotDetectionCore}
\subsection{Implementierung}
\subsection{Scoringlogik}

\newpage
\section{Projektrealisierung}
\subsection{Entwickelte Komponente}
\subsection{Projektstruktur}
\subsection{Implementierungsszenario}
\subsubsection{OwnVehicleInformation}
\subsubsection{SensorDataHandler}
\subsubsection{CamHandler}
\subsubsection{BlindSpotDatection}
\subsubsection{HmiConnector}

\newpage
\section{Projektvalidierung}
\subsection{Testrahmen}
\subsection{Testszenarien}
\subsubsection{Die Teststrecke}
\subsubsection{Zenarien}
\subsubsection{Testablauf mit Bilder}

\newpage
\section{Fazit}
\section{Literatur}
\thispagestyle{empty}

\begin{thebibliography}{9}
\bibitem[1]{Kurs1} \emph{
Handbuch Fahrerassistenzsysteme
Grundlagen, Komponenten und Systeme für aktive Sicherheit und Komfort},
Hermann Winner, Stephan Hakuli, Felix Lotz, Christina Singer,3 Auflage, 2015. URL: https://link.springer.com/book/10.1007/978-3-658-05734-3
 
\bibitem [Frank 05]{kurz1}\emph{https://felix.apache.org/documentation/subprojects/apache-felix-config-admin.html}
\end{thebibliography}

\section{Abkürzungen}
\end{document}
\endcsname 
